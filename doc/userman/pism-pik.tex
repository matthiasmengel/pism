

\section{PISM-PIK improvements for marine ice sheet modeling}
\label{sec:pism-pik}
\index{PISM!PISM-PIK}
\optsection{PISM-PIK}

References \cite{Albrechtetal2011} and \cite{Winkelmannetal2011} describe improvements to the grounded, SSA-as-a-sliding law model of \cite{BBssasliding}.  These improvements make PISM an effective Antarctic model, as demonstrated by \cite{Martinetal2011,Levermann2011}.  Because these improvements had a separate existence as the PISM-PIK model from 2009--2010, we call them the ``\emph{PISM-PIK improvements}'' here.

These model improvements all relate to the stress balance and geometry changes that apply at the vertical calving face of floating ice shelves.  The physics at such calving fronts is very different from elsewhere on an ice sheet, because the flow is nothing like the lubrication flow addressed by the SIA, and nor is the physics like the sliding flow in the interior of an ice domain, where sliding alters geometry according to the basic mass continuity equation.  The correct physics at the calving front can be thought of as certain modifications to the mass continuity equation and to the SSA stress balance equation.  The code implementing the PISM-PIK improvements makes these highly-nontrivial modifications of the finite difference/volume equations in PISM.

\subsection{Partially-filled cells at the boundaries of ice shelves}
\label{sec:part-grid}

Albrecht et al \cite{Albrechtetal2011} argue that the correct movement of the ice shelf calving front on a finite-difference grid, assuming for the moment that ice velocities are correctly determined (see below), requires tracking some cells as being partially-filled (option \intextoption{part_grid}). If the calving front is moving forward, for example, it is not correct to add a little mass to the next cell as if that next cell was filled with a thin layer of ice (which would smooth the steep ice front after a couple of time steps).  The PISM-PIK mechanism adds mass to the partially-filled cell which the advancing front enters and determines the coverage ratio according to the ice thickness of neighboring fully-filled ice shelf cells.  If option \texttt{-part_grid} is used then the PISM output file will have field \texttt{Href} which shows the amount of ice in the partially-filled cells as a thickness.  When a cell becomes fully-filled in the sense that the \texttt{Href} thickness equals the average of neighbors then the residual mass is redistributed to neighboring partially-filled or empty grid cells if option \intextoption{part_redist} is set.

The equations determining the velocities are only sensitive to ``fully-filled'' cells. Advection is controlled by values of velocity in fully-filled cells. Adaptive time stepping (CFL criterion) limits the speed of ice front propagation.

A summary of PISM options to turn on these PISM-PIK mechanisms is in Table \ref{tab:pism-pik-part-grid}.

\begin{table}[ht]
  \centering
 \begin{tabular}{lp{0.7\linewidth}}
    \\\toprule
    \textbf{Option} & \textbf{Description}
    \\\midrule
    \intextoption{part_grid} & allows the ice shelf front to advance by a part of a grid cell, avoiding
	the development of unphysically-thinned ice shelves\\
    \intextoption{part_redist} &  scheme which makes the -part_grid mechanism conserve mass\\ 
    \intextoption{cfbc} & applies the stress boundary condition along the ice shelf calving front.\\
    \intextoption{kill_icebergs} & identify and eliminate free-floating icebergs, which cause well-posedness problems for the SSA stress balance solver \\
    \midrule
    \intextoption{pik} & equivalent to option combination ``\texttt{-cfbc -kill_icebergs -part_grid -part_redist}'' \\
    \bottomrule
 \end{tabular}
\caption{Options which turn on PISM-PIK ice shelf front mechanisms.}
\label{tab:pism-pik-part-grid}
\end{table}


\subsection{Stress condition at calving fronts}
\label{sec:cfbc}
The vertically integrated force balance at floating calving fronts has been formulated by \cite{Morland} as
\begin{equation}
\int_{z_s-\frac{\rho}{\rho_w}H}^{z_s+(1-\frac{\rho}{\rho_w})H}\mathbf{\sigma}\cdot\mathbf{n}\;dz = \int_{z_s-\frac{\rho}{\rho_w}H}^{z_s}\rho_w g (z-z_s) \;\mathbf{n}\;dz.
\label{MacAyeal2}
\end{equation}
with $\mathbf{n}$ being the horizontal normal vector pointing from the ice boundary oceanward, $\mathbf{\sigma}$ the \emph{Cauchy} stress tensor, $H$ the ice thickness and $\rho$ and $\rho_{w}$ the densities of ice and seawater, respectively, for a sea level of $z_s$. The integration limits on the right hand side of Eq.~\eqref{MacAyeal2} account for the pressure exerted by the ocean on that part of the shelf, which is below sea level (bending and torque neglected). The limits on the left hand side change for water-terminating outlet glacier or glacier fronts above sea level according to the bed topography. Applying the ice flow law (Sect.~\ref{sec:rheology}) Eq.~\eqref{MacAyeal2} can be written in terms of strain rates (velocity derivatives).

Note that the discretized SSA stress balance, in the default finite difference discretization chosen by \intextoption{ssa_method} \texttt{fd}, is solved with an iterative matrix scheme. During matrix assembly, those grid cells along the ice domain boundary (fully-filled) are replaced according to Eq.~\eqref{MacAyeal2} to apply the correct forces, when option \intextoption{cfbc} is set.  Details can be found in \cite{Winkelmannetal2011} and \cite{Albrechtetal2011}.  

\subsection{Calving}
\label{sec:calving}
\optsection{Calving}
At the start, PISM-PIK included a physically based 2D-calving parameterization, further developed in \cite{Levermannetal2012}. This calving parameterization is turned on by option \intextoption{eigen_calving}.  Average calving rates, $c$, are proportional to the product of principal components of the horizontal strain rates, $\dot{\epsilon}_{_\pm}$, derived from SSA-velocities 
\begin{equation}
\label{eq: calv2}
c = K\; \dot{\epsilon}_{_+}\; \dot{\epsilon}_{_-}\quad\text{and}\quad\dot{\epsilon}_{_\pm}>0\:.
\end{equation}
The constant $K$ incorporating material properties of the ice at the front can be set using the \intextoption{eigen_calving_K} option or a configuration parameter (\texttt{eigen_calving_K} in \texttt{src/pism_config.cdl}).

The actual strain rate pattern strongly depends on the geometry and boundary conditions along the the confinements of an ice shelf (coast, ice rises, front position).  The strain rate pattern provides information in which regions preexisting fractures are likely to propagate, forming rifts (in two directions).  These rifts may ultimately intersect, leading to the release of icebergs. This and other ice shelf model calving models are not intended to resolve individual rifts or calving events. This first-order approach produces structurally-stable calving front positions which agree well with observations.  Calving rates balance terminal SSA velocities on average.

The partially-filled grid cell formulation (subsection \ref{sec:part-grid}) provides a framework suitable to relate the calving rate produced by \intextoption{eigen_calving} to the mass transport scheme at the ice shelf terminus.  Ice shelf front advance and retreat due to calving are limited to a maximum of one grid cell length per (adaptive) time step.

PISM also includes three more basic calving mechanisms (Table \ref{tab:calving}). The option \intextoption{thickness_calving} is based on the observation that ice shelf calving fronts are commonly thicker than about 150--250\,m (even though the physical reasons are not clear yet). Accordingly, any floating ice thinner than $H_{\textrm{cr}}$ is removed along the front, at a rate at most one grid cell per time step. The value of $H_{\mathrm{cr}}$ can be set using the \intextoption{calving_at_thickness} option or the \texttt{calving_at_thickness} configuration parameter.

Option \intextoption{float_kill} removes (calves) any ice that satisfies the flotation criterion; this option means there are no ice shelves in the model at all.

Option \fileopt{ocean_kill} reads in the ice thickness field from a file. Any locations which were ice-free ocean in the provided data set are places where floating ice is removed. By omitting the file name it is possible to use the ice thickness at the beginning of the run. When 

\begin{table}[ht]
  \centering
  \begin{tabular}{lp{0.6\linewidth}}
    \toprule
    \textbf{Option} & \textbf{Description} \\
    \midrule
    \intextoption{eigen_calving} & Physically-based calving parameterization \cite{Levermannetal2012,Winkelmannetal2011}.  Calving proportional to product of principle strain rates, where they are positive. \\
    \intextoption{eigen_calving_K} ($K$) & sets the calving parameter $K$ \\
    \intextoption{thickness_calving} & enables grid-cell wise calving depending on terminal ice thickness \\
    \intextoption{calving_at_thickness ($H_{\textrm{cr}}$)} & sets the terminal thickness threshold \\
    \intextoption{float_kill} & All floating ice is calved off immediately.\\
    \fileopt{ocean_kill} & All ice flowing into grid cells marked as ``ice free ocean'' according to the ice thickness in a provided file \\
    \bottomrule
  \end{tabular}
\caption{Calving options}
\label{tab:calving}
\end{table}


\subsection{Iceberg removal}
\label{sec:kill-icebergs}
The PISM-PIK calving mechanism removes ice along the seaward front of the ice shelf domain. This can lead to isolated grid cells of floating (or partially filled) ice or even a patch of floating ice (iceberg) fully surrounded by ice free ocean neighbors and hence detached from the feeding ice sheet. In such a situation the stress balance solver in SSA is not well-posed \cite{SchoofStream}. Option \intextoption{kill_iceberg} cleans this up.  It identifies such regions by checking iteratively for grid neighbors that are grounded, creating an ``iceberg mask'' showing floating ice that is, and is not (icebergs), attached to grounded ice.  It then eliminates these free-floating icebergs.


%%% Local Variables:
%%% mode: latex
%%% TeX-master: "manual"
%%% End:

% LocalWords:  html PISM PISM's
